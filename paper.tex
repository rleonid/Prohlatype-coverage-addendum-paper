% Prohlatype Coverage Addendum

% <- General comment to be ignored while reading the text of the paper.

% \begin{comment}
% Potential text that isn't included at the moment.
% \end{comment}

%\documentclass[twocolumn]{article}
\documentclass{article}
\usepackage{authblk}          % For 2+ authors

\newcommand{\RNum}[1]{\uppercase\expandafter{\romannumeral #1\relax}}

\begin{document}


\title{A Coverage Likelihood for HLA Typing.}
\author[1]{Leonid Rozenberg\thanks{leonid.rozenberg@mssm.edu}}
\author[2]{Andrew Kasarskis\thanks{ TODO }}
\affil[1]{Department of Genetics and Genomic Sciences, Icahn School of Medicine at Mount Sinai, New York, New York, USA }
\affil[2]{Department of Genetics and Genomic Sciences, Icahn School of Medicine at Mount Sinai, New York, New York, USA }
%\date{\today}
\maketitle

% The breakdown of sections is roughly based on PLOS comp-bio
% http://journals.plos.org/ploscompbiol/s/submission-guidelines

\begin{abstract}

  We describe a novel likelihood based upon how reads \emph{cover}
  a loci to facilitate calculating accurate HLA genotype posteriors.

\end{abstract}

\section{Introduction}
% The Introduction should put the focus of the manuscript into a broader context. As you compose the Introduction, think of readers who are not experts in this field. Include a brief review of the key literature. If there are relevant controversies or disagreements in the field, they should be mentioned so that a non-expert reader can delve into these issues further. The Introduction should conclude with a brief statement of the overall aim of the experiments and a comment about whether that aim was achieved.

% What is HLA-typing?

See first paper.

\section{Results}
% The Results section should provide details of all of the experiments that are required to support the conclusions of the paper. There is no specific word limit for this section, but details of experiments that are peripheral to the main thrust of the article and that detract from the focus of the article should not be included. The section may be divided into subsections, each with a concise subheading. The section should be written in the past tense.

Math goes here.

\section{Implementation}

\section{Conclusions}

It resolves that bad C case.

\clearpage

\bibliographystyle{plain}
\bibliography{paper}

\clearpage

\appendix

\end{document}
